\documentclass[]{article}
\usepackage{a4wide}

\begin{document}

\title{Review: Efficient Broadcast on Fragmented Spectrum in Cognitive Radio Networks}
\author{Freek van Tienen \\ 4094123}
\date{22-03-2016}
\maketitle

This paper indicates a method to efficiently broadcast in Fragmented Spectrum, we start with the broadcast intention.
In effort to make this work it uses an $N$-point IFFT to encode a single OFDM random intentation sequence and the broadcaster transmits this with a cyclic prefix of itself in all available channels for the transmitter.
The receiver then performs an $N$-point FFT each time a new sample is obtained and uses convolution to detect spikes and with that a possible broadcast.
It will then decode the unicast sequence of the broadcaster and look it up the the neighbour table.
The receiver will then transmit his own unicast sequence after $priority * 2$ QFDM symbols on all available channels for him.
When the broadcaster received all the unicast sequences it  uses a greedy algorithm to select the best channel combination scheme that provides maximum channel width.
It then communicates this by adding up all the unicast sequences for the specific channel and transmits this.

To analyse the effectiveness of the convolution-based spectrum agreement they use the GNU Radio / USRP platform.
To show the advantages of SFAB they use NS-2 to simulate how narrowband interference affects both media access opportunities at the sender and the data receiving at the receivers.

\section*{Review}
One of the main things that is missing in the paper is the scalability of this broadcasting protocol.
The simulations in NS-2 are only done with 10 wide-band devices and the tests with USPR only with 1 broadcaster and 1 receiver. 

The measurements are only showing averages, but no variance or standard deviation.
There is also no information about the repeatability and amount of test runs of the experiments in NS-3 and USRP.
Because also no source code is provided, people cannot confirm these test results.

The comparison between 802.11 and spectrum-agile only shows the advantages of SFAB, but no tests where shown in non-interference situations.
Also no statements where made about the false detection of broadcast intentions or unicast sequences or not detecting them.

Next to that the setup phase of the nodes and the selection of unicast sequences is left out of the paper, which doesn't show how usable the system really is.
No information is given on how many unicast sequences can be generated and how they interfere with each other when they are convoluted.

\end{document}